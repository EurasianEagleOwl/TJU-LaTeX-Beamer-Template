\documentclass[10pt,aspectratio=43,table,fontset=none]{beamer} 
%设置为 Beamer 文档类型,设置字体为 10pt,长宽比为16:9,数学字体为 serif 风格
\batchmode
\usepackage{graphicx}
\usepackage{animate}
\usepackage{hyperref}

%导入一些用到的宏包
\usepackage{amsthm,amssymb,amsmath,bm,amsfonts,enumerate,epsfig,bbm,calc,color,ifthen,capt-of,multimedia,hyperref}
\usepackage{mathrsfs}
\usepackage{mathptmx}
\usepackage{amsfonts}
\usepackage{xeCJK} %导入中文包
\usepackage{indentfirst}
%\setlength{\parindent}{2em}         %开头空两格(可选)







\usetheme[compress]{Berlin} %主题
\usecolortheme{tju} %主题颜色

\usepackage[ruled,linesnumbered]{algorithm2e}

\usepackage{fancybox}
\usepackage{xcolor}
\usepackage{times}
\usepackage{listings}

\usepackage{booktabs}
\usepackage{colortbl}
\usepackage{caption}
\setbeamertemplate{caption}[numbered]

\usepackage{fontspec}                  %引入字体设置宏包
\newfontfamily\kaisu{STKaiti}       %定义华文楷体为\kaisu
\setsansfont{TeX Gyre Termes}       %设置西文字体为times new roman
\setCJKsansfont{SimSun}             %设置中文字体为宋体
\setCJKmonofont{STKaiti}
\setmonofont{TeX Gyre Termes}   
\setbeamerfont{frametitle}{family=\ttfamily}


\newcommand{\Console}{Console}
\lstset{ %
	backgroundcolor=\color{white},   % choose the background color
	basicstyle=\footnotesize\rmfamily,     % size of fonts used for the code
	columns=fullflexible,
	breaklines=true,                 % automatic line breaking only at whitespace
	captionpos=b,                    % sets the caption-position to bottom
	tabsize=4,
	commentstyle=\color{mygreen},    % comment style
	escapeinside={\%*}{*)},          % if you want to add LaTeX within your code
	keywordstyle=\color{blue},       % keyword style
	stringstyle=\color{mymauve}\ttfamily,     % string literal style
	numbers=left, 
	%	frame=single,
	rulesepcolor=\color{red!20!green!20!blue!20},
	% identifierstyle=\color{red},
	language=python
}


\definecolor{mygreen}{rgb}{0,0.6,0}
\definecolor{mymauve}{rgb}{0.58,0,0.82}
\definecolor{mygray}{gray}{.9}
\definecolor{mypink}{rgb}{.99,.91,.95}
\definecolor{mycyan}{cmyk}{.3,0,0,0}
%\setbeamertemplate{navigation symbols}{}
\renewcommand{\tablename}{表}                                     % 插表中文题头
\renewcommand{\figurename}{图}                                 % 插图中文题头
\renewcommand{\lstlistingname}{代码}
\renewcommand{\lstlistlistingname}{代码汇总}                     %代码中文题头
%\renewcommand{\thefigure}{\arabic{section}-\arabic{figure}}       % 使图编号为 7-1 的格式 %\protect{~}
%\renewcommand{\thetable}{\arabic{section}-\arabic{table}}         % 使表编号为 7-1 的格式
%题目,作者,学校,日期
\title{Reinventing the Wheel: Publishing High-quality Slides}
\subtitle{\fontsize{9pt}{14pt}\textbf{平均场博弈论的计算方法}}
\author{答辩人: John  \newline \quad 指导老师: Good Joe}
\institute{天津大学数学学院18级求是数学班}
\date{\today}
%学校Logo
\pgfdeclareimage[height=1cm]{tju-logo}{figures/tju-logo.eps}
\logo{\pgfuseimage{tju-logo}\hspace*{-0.1cm}\vspace*{-0.25cm}}

%\AtBeginSection[]
%{
%	\begin{frame}<beamer>
%	\frametitle{\textbf{目录}}
%	\tableofcontents[currentsection]
%\end{frame}
%}
\beamerdefaultoverlayspecification{<+->}
% -----------------------------------------------------------------------------
\begin{document}
% -----------------------------------------------------------------------------
\begin{frame}
	\titlepage
\end{frame}

\section[目录]{}   %目录
\begin{frame}{目录}
\tableofcontents
\end{frame}

% -----------------------------------------------------------------------------
\section{引言}  %引言
\subsection{研究背景}
\begin{frame}{研究背景}
	\begin{columns}[T] % align columns
		\begin{column}<0->{.4\textwidth}
			\begin{figure}[htpb!]
				\centering
				\resizebox{1\linewidth}{!}{
					\includegraphics{figures/TJUMath.png}
				}
				%\includegraphics[scale=1.0]{figurefile}
				\caption{天津大学数学学院}
				\label{fig:campus}
			\end{figure}
		\end{column}%
		\hfill%
		\begin{column}<0->{.6\textwidth}
			\begin{itemize}
				\item<1-> 平均场指个体数趋于无穷且彼此不可区分的情况
				\begin{itemize}
					\item<1-> 本文主要研究平均场思想下的非合作博弈(平均场博弈论,简称MFG)的Nash均衡
				\end{itemize}
				\item<1-> 合作博弈(平均场控制论,简称MFC)的最优控制
				\begin{itemize}
					\item<1-> 及其数值计算方法。给出MFG和MFC的动态系统和成本函数后
				\end{itemize}
			\end{itemize}
		\end{column}%
	\end{columns}
\end{frame}
\subsection{主要工作}
\begin{frame}{主要工作}
完成这项工作需要如下步骤
\begin{block}{具体步骤}
\begin{itemize}
	\item<0->  对SMS数据进行迄今为止最大的挖掘分析
	\item<0->  评估良性短消息服务的安全态势
	\item<0->  刻画通过SMS网关进行的恶意行为
\end{itemize}
\end{block}
\end{frame}

 \begin{frame}
\frametitle{OTT服务}
\begin{figure}[!t]
	\centering
	\includegraphics[width=2in]{figures/TJUMath.png}
	\caption{OTT服务}
	\label{figure3_OTT}
\end{figure}
\begin{equation}
	\left\{
	\begin{aligned}
		&\frac{\partial m^{\hat{m},\alpha}}{\partial t}(t,x)-\nu\Delta m^{\hat{m},\alpha}(t,x)+\text{div} \left(m^{\hat{m},\alpha}(t,\cdot)b(\cdot,\hat{m}(t),\alpha(t,\cdot))\right)(x)=0,\ \forall(t,x)\in\left(0,T\right]\times\mathbb{T}^d\\
		&\ m^{\hat{m},\alpha}(0,x)=m_0(x),\ \forall x\in\mathbb{T}^d.
	\end{aligned}
	\right.\label{3-1}
\end{equation}

\end{frame}



\section{词表示模型}  %自我介绍

\begin{frame}{词表示}
在NLP任务中,可以利用各种词表示模型,将“词”这种符号信息表示成数学上的向量形式。。将语义信息表示成稠密、低维的实值向量,这样就可以用计算向量之间相似度的方法(如余弦相似度),来计算语义的相似度。词的向量表示可以作为各种深度学习模型的输入来使用
\begin{block}{词表示模型分类}
	直接表示模型
	\begin{itemize}
		\item<0-> One-Hot Representation
	\end{itemize}
	
	分布式表示模型
	\begin{itemize}
		\item<0-> 计数模型(基于共现矩阵)
		\item<0-> 预测模型(基于神经网络)
	\end{itemize}
\end{block}
\end{frame}


\section{直接表示模型}
\begin{frame}{One-Hot Representation}
最简单直接的词表示是One-Hot Representation。考虑一个词表$ \mathbb V $,里面的每一个词$ w_i $都有一个编号$ i\in \{1,...,n\} $,那么词$ w_i $的one-hot表示就是一个维度为n的向量,其中第$ i $个元素值非零,其余元素全为0。例如:
\[  w_2=[0,1,0,...,0]^\top  \]
\[  w_3=[0,0,1,...,0]^\top  \]
\begin{block}{缺点}
	\begin{itemize}
		\item<0-> 彼此正交,不能反应词间的语义关系
		\item<0-> 稀疏表示,维度很高,和词典大小成正比
	\end{itemize}
\end{block}
\begin{center}
	\textcolor{mymauve}{仅仅是为了区分词,不包含语义信息,语义信息应该从上下文中挖掘}
\end{center}
\end{frame}


\section{研究方法与数据集特征}
\begin{frame}{研究方法与数据集特征}
\begin{columns}[c] % align columns
	\begin{column}<0->{.5\textwidth}
		\vspace*{1cm}
		\begin{itemize}
			\item 使用Scrapy框架爬取公共网关
		\end{itemize}
	
		\begin{itemize}
			\item 收集8个公共短信网关在14个月的数据
		\end{itemize}
	
		\begin{itemize}
			\item 共抓取386,327条数据
		\end{itemize}
    \end{column}%
\hfill%	
	\begin{column}<0->{.40\textwidth}
		\begin{table}
			\caption{公共网关抓取的信息数}
			\footnotesize
			\rowcolors{1}{mygray}{white}
			\begin{tabular}{|c|c|}
				\hline
				\textbf{Site}           & \textbf{Messages}\\
				\hline
				receivesmsonline.net    &81313\\
				\hline
				receive-sms-online.info &69389\\
				\hline
				receive-sms-now.com     &63797\\
				\hline
				hs3x.com               &55499\\
				\hline
				receivesmsonline.com    &44640\\
				\hline
				receivefreesms.com      &37485\\
				\hline
				receive-sms-online.com  &27094\\
				\hline
				e-receivesms.com       &7107\\
				\hline
			\end{tabular}
		\end{table}
    \end{column}%
\end{columns}
\end{frame}

\begin{frame}
\frametitle{消息聚类分析}
\begin{block}{\textbf{基本思路}}
	\begin{itemize}
		\item<0-> 使用编辑距离矩阵将类似的消息归于一张连通图中。
		\item<0-> 使用固定值替换感兴趣的消息,如代码、email地址。
		\item<0-> 查找归一化距离小于阈值的消息,并确定聚类边界。
	\end{itemize}
\end{block}

\begin{block}{\textbf{实现步骤}}
	\begin{enumerate}
		\item<0-> 加载所有消息。
		\item<0-> 用固定的字符串替换数字、电子邮件和URL以预处理消息。
		\item<0-> 将预处理后的信息按字母排序。
		\item<0-> 通过使用编辑距离阈值(0.9)来确定聚类边界。
		\item<0-> 手动标记各个聚类,以确定服务提供者、消息类别等。
	\end{enumerate}
\end{block}
\end{frame}

\section{算法和代码}
\subsection{算法}
\begin{frame}{算法}
\begin{algorithm}[H]
	\caption{HOSVD}
	\small 
	\KwIn{HOSVD($\mathcal{X},R_{1},R_{2}.....R_{N}$) }
	\KwOut{ $\mathcal{G},A_{(1)},A_{(2)}......A_{(N)} $ }
	
	\For{$k=1$ to $N$ }
	{
		$A_{(n)}\leftarrow R_{n}$left singular matrix of $X_{(n)}$
	}
	$\mathcal{G}=\leftarrow \mathcal{X} \times A_{(1)}^{T} \times A_{(2)}^{T}...... \times A_{(N)}^{T}$\\
	\Return $\mathcal{G},A_{(1)},A_{(2)}......A_{(N)} $
\end{algorithm}
\end{frame}

\subsection{代码}
\begin{frame}[fragile]{代码}
HOSVD在Python的代码实现和分析:
\lstinputlisting[lastline=11,
language=Python,
frame=single,
caption=First ten lines of some Python code,
label=python]
{HOSVD.py}
\end{frame}


\section{Future Work}
\begin{frame}{Future Work}  %将来可做的方向
\begin{itemize}
\item<0-> Get more people to try this
\item<0-> Benchmark the entire system in the wild
\item<0-> Profit!
\end{itemize}
\end{frame}

\begin{frame}{Thank you}
\begin{center}
\begin{minipage}{1\textwidth}
	\setbeamercolor{mybox}{fg=white, bg=black!50!blue}
 \begin{beamercolorbox}[wd=0.70\textwidth, rounded=true, shadow=true]{mybox}
\LARGE \centering Thank you for listening!  %结束语
\end{beamercolorbox}
 \end{minipage}
\end{center}
\end{frame}

\begin{frame}{Q\&A}
\begin{center}
	\begin{minipage}{1\textwidth}
		\setbeamercolor{mybox}{fg=white, bg=black!50!blue}
		\begin{beamercolorbox}[wd=0.70\textwidth, rounded=true, shadow=true]{mybox}
			\LARGE \centering  Questions?  %请求提问
		\end{beamercolorbox}
	\end{minipage}
\end{center}
\end{frame}

% -----------------------------------------------------------------------------
\end{document}
%文档结束